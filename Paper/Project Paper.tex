%%
%%                  TEMPLATE for Math 436 project report
%%
\documentclass[11pt]{amsart}
%%% WARNING: Do NOT change the page size, fonts, or margins!  Penalties will apply.


% \usepackage{chemformula}
\usepackage[version=4]{mhchem}  % Used for chemical equations
\usepackage{graphicx}
\usepackage{amssymb,amsmath,amsthm}
\usepackage{placeins} %enables \FloatBarrier, which prevents figures and tables from going below it.
\usepackage{hyperref} %makes cross references into hyperlinks. 
\graphicspath{ {./images/} }


%%% WARNING: Do NOT change the page size, fonts, or margins!  Penalties will apply.
%%% WARNING: Do NOT change the page size, fonts, or margins!  Penalties will apply.

\begin{document}

\title{Modeling Yeast Spread in Bread Dough}
\author{Connor McBride, Ethan Palenske, Jackson Pond}

% Change the date to match the date you actually wrote this paper
\date{19 November 2025} % or use \today

\begin{abstract}
Place abstract here. The abstract summarizes in one paragraph the main question and conclusions draw from your investigation.
\end{abstract}

\maketitle % this actually makes the title

%% First Section
\section{Background/Motivation}

\par The primary question this paper seeks to answer is determining the sorts of models that accurately match the chemical reactions that yeast undergo within bread. Yeast interacts in complicated ways with the other ingredients in bread, and even with its own byproducts. We also want to verify if existing SIR and other pandemic models can accurately mimic the spread of yeast, and if not, what factors make these models no longer accurate. Determining the optimal initial conditions (quantities of yeast, water, etc.) and how these affect the growth is a secondary focus of this paper.

\par \textbf{TALK ABOUT IMPACT}

% \begin{figure}[htb]
% \begin{center} %Put your images in a figure like this
% \includegraphics[width=\textwidth]{Myfig.pdf} % Better to make them pdfs than png or gif or jpeg
% \end{center}
% \caption{Plots should be high resolution (pdf 300dpi), uncluttered, a reasonable size, and easily readable and understandable.  All figures should have a complete caption that helps the reader make sense of the figure even if they haven't read the paper yet. Here is an example: This figure shows the risk or mean squared error (in black) of for a generic family of regression models as a function of model complexity (the number of free parameters in the model).  The generalized aliasing decomposition shows that the risk is the sum of three parts: Model insufficiency (red), Data insufficiency (green), and Aliasing (blue).  Model insufficiency is monotonically decreasing as a function of complexity, Data insufficiency vanishes when the number of parameters is less than the number of training points (the classical regime), but increases monotonically in the modern regime (where the number of parameters is greater than the number of data points).  Aliasing generally increases up to the interpolation threshold and decreases thereafter, converging to zero almost surely as the number of parameters goes to infinity.  
% }
% \label{fig:MeanSquaredError} % for automatic cross referencing
% \end{figure}


\subsection*{The Chemistry of Bread}
\;
\par There are four primary ingredients in bread dough: flour, water, salt, and yeast. Flour provides not only the gluten that gives bread its structure, but also contains sugars that the yeast consumes. Salt strengthens the bread dough, making it more elastic.

\par Once water is mixed with flour, enzymes from the flour activate, including alpha-amylase and beta-amylase. These enzymes break up the long starch chains in the flour, quickly converting these into glucose and maltose. These sugars, unlike starch itself, are edible by yeast. \cite{bread_chemistry_1}

\par Yeast is a type of single-celled fungus that produces the \ce{CO_2} that forms bubbles in bread. This process is called fermentation, and yeast undergoes two primary phases within bread. The first is aerobic respiration, where yeast consumes oxygen and sugar and produces carbon dioxide and ATP, which is a molecule that stores energy for cells.
\[
\ce{6O_2} + \ce{C_6H_{12}O_6} \to \ce{6CO_2} + \ce{6H_2O} + \ce{ATP}.
\]

\par In the second phase, called anaerobic respiration, occurs once the oxygen in the bread is exhausted. Yeast consumes glucose and begins producing ethanol instead of water as a biproduct. The amount of ethanol produces is very minimal compared to alcohol fermentation, and evaporates during the baking process. The yeast still produces \ce{CO_2} as a byproduct, but it produces it at a slower rate.
\[
\ce{C_6H_{12}O_6} \to \ce{2CO_2} + \ce{2C_2H_5OH} + \ce{ATP}. \text{\cite{bread_chemistry_2}}
\]
\cite{bread_chemistry_2}

\par Some of the more advanced models in this paper sought to incorporate this two-phase structure.

\par \textbf{WE NEED TO MENTION THE MODELS / TECHNIQUES THAT WERE USED BEFORE US TO MODEL THIS PROBLEM}


%% Second Section 
\section{Modeling}

The primary aspect of the project is the modeling of the chosen phenomenon. If your group's repeated attempts resulted in abject failure, or your group succeeded, detail them in this section. Be sure to account for the various attempted models and why they were not appropriate. Include numerical simulations for each attempted model. Reference figures and plots, like Figure~\ref{fig:MeanSquaredError}.

\subsection*{Simple Compartmental Model}

\par This was the simplest of the models we implement. The goal was to get a baseline for future models and to see if we can fit real-world data with a simple compartmental model. If this did not fit data well, we at least hoped that the model would be qualitatively accurate, in that the amount of yeast and byproducts increase over time while the amount of sugar decreases. We used sugar, yeast, and byproducts as our three compartments. The model's differential equation is given by
\begin{align*}
    \dot S &= -\alpha S Y\\
    \dot Y &= \beta S Y - \gamma Y\\
    \dot B &= \delta S Y,
\end{align*}
where $\alpha, \beta, \gamma$ and $\delta$ are constants. In particular, $\alpha$ controls the quantity of sugar eaten by yeast in each interaction, $\beta$ controls the reproduction rate of yeast as it consumes sugar, $\gamma$ controls the death rate of yeast, and $\delta$ controls the rate at which byproducts are produced as yeast consumes sugar.


\includegraphics[scale=0.5]{images/simple_compartmental.png}

\par 

alpha: 0.1127983210116926
beta: 0.017788187881001945
gamma: 0.01
delta: 0.04748943834320091

\par Due to the simplicity of this model, we do not attempt to justify values of the constants through physical explanations. - \textbf{CHANGE THIS! ESTIMATE WHY THESE VALUES WORK}

\subsection*{Compartmental Model with Monod Equation}
We now seek to incorporate the Monod equation into a compartmental model. The Monod equation models the growth of a microorganism based on the concentration of a limiting substance. \cite{monod_1} The standard form of the Monod equation is given by 
\[
    \mu = \mu_{\text{max}} \frac{S}{K_s + S},
\]
where $\mu$ is the growth rate of a microorganism, $\mu_{\text{max}}$ is the maximum growth rate of the microorganism, $S$ is the concentration of the limiting substance for growth, and $K_s$ is the ``half-velocity'' constant (the value of $S$ when $\mu / \mu_{\text{max}} = 0.5)$. \cite{monod_2}

Some modifications to a traditional SIR model were required in order to fit the Monod equation into a compartmental model. SIR models typically assume a closed population, but for modeling yeast fermentation, compartments don't simply flow into one another as some mass escapes the system. 
Similarly, the yeast growth and fermentation are based on growth rates and yield coefficients, not just population flow. Let $Y$ depict the quantity of yeast, $S$ denote the quantity of sugar, and $P$ denote the byproducts produced (e.g. ethanol and $\ce{CO_2}$). The monod categorical model is given by 
\begin{align*}
    \dot{Y} &= \mu_{\text{max}} \frac{S}{K_s + S}Y \\
    \dot{S} &= -\frac{1}{Y_{X/S}} \dot{Y} \\
    \dot{P} &= Y_{P/S} (-\dot{S}),
\end{align*}
where $Y_{X/S}$ denotes the grams of biomass of yeast per gram of sugar, and $Y_{P/S}$ denotes the product yield of yeast per gram of sugar. Plotting for $t \in [0, 10]$, the plot of the solution is given by 
\begin{figure}[htbp]
    \centering
    \includegraphics[scale=0.6]{monod_categorical.png}
    \caption{Monod categorical model}
    \label{fig:monod_categorical}
\end{figure}
We use the initial conditions $(Y_0, S_0, P_0) = (0.1, 10, 0)$ (each with units g/L). We also suppose that $K_s = 0.6$, $Y_{X/S} = 0.1$, and $Y_{P/S} = 1$. 

\subsection*{KM Mechanistic}

\subsection*{MacArthur Consumer-Resource}
\par The previous models make the following simplifying assumptions: 1) That all sugar is initially available to be consumed by yeast and only decreases over time, 2) that the rate and efficacy of yeast's consumption of sugar depends only on the respective amounts of yeast sugar, and 3) that oxygen is not a resource on which yeast feeds.

\par In this section, we attempt to incorporate more realistic assumptions into our model and better approximate the real dynamics of a rising loaf of dough. A more appropriate family of models to use would be Consumer-Resource models, which are designed to factor in multiple consumers and multiple replenishing resources. A special case of these models are the basic Lotka-Volterra predator-prey models, which have a single consumer and a single resource. The Consumer-Resource family of models considers the replenishment of the resource, predators' affinity for different resources (called uptake in this paper), and the yield in biomass for the consumer per unit of resource consumed. In the case considered by this paper, there is a single consumer (yeast), and two resources (sugar and oxygen). Additionally, both the uptake and yield of sugar vary with the amount of oxygen, as we will discuss.

\par The general form of the MacArthur consumer-resource model for S species of consumers with concentrations $N_i$ for $i \in {1, 2, ..., S}$, and M resources with concentrations $R_{\alpha}$ for $\alpha \in {1, 2, ..., M}$ is
\begin{align}
\frac{dN_i}{dt} &= N_i \sum_\alpha (u_{i\alpha}(R_{\alpha}) w_{i\alpha}) - m_i \\
\frac{dR_\alpha}{dt} &= F_{\alpha} (R_{\alpha}) - \sum_i (N_i u_{i\alpha}(R_{\alpha}))
\end{align}

\par $u_{i\alpha}$ controls the uptake of the resource by the consumer, or how quickly consumer $i$ consumes resource $\alpha$ per unit biomass. This can vary with the concentration of the resource, as it can saturate \cite{AdaptiveStrategies}, or we can keep it a simple constant. $w_{i\alpha}$ is the *yield* that resource $\alpha$ provides consumer $i$, and controls the proportion between resource $\alpha$ consumed and biomass added to consumer $i$. $m_i$ is the natural mortality rate of consumer $i$. $F_{\alpha}$ is the function that controls the replenishment of resource $\alpha$ in the absence of consumers.

\par The amount of parameters and associated dimensions can be disorienting. We will provide a sort of index here of the parameters we will use in our model, and then we will detail the way each fits into our model.

\begin{table}[h!]
\centering
\begin{tabular}{l c c c}
\hline
\textbf{Parameter} & \textbf{Value} & \textbf{Units} & \textbf{Notes} \\ 
\hline
$Y$ & -- & $\frac{g_Y}{L}$ & Biomass of dry yeast per liter of dough \\
$O$ & -- & $\frac{mmol}{L}$ & Count of $O_2$ per liter of dough \\
$S$ & -- & $\frac{GGE}{L}$ & Grams of Glucose-Equivalent per liter of dough \\
$C$ & -- & $\frac{mol}{L}$ & Count of $CO_2$ per liter of dough \\
$K_O$ &  & $\frac{mmol}{L}$ & Half saturation for Oxygen \\
$K_S$ & & $\frac{GGE}{L}$ & Half saturation for Sugar \\
$P_S$ & & $\frac{g_S}{L}$ & Carrying capacity for sugar \\
$r_S$ & & $\frac{1}{h}$ & Sugar logistic growth rate \\
$f(O)$ & & $dimensionless$ & Oxygen availibility switch \\
$w_S^{aer}$ & & $\frac{g_Y}{g_S}$ & aerobic ATP yield, very efficient
$w_S^{ana}$ & & $\frac{g_Y}{g_S}$ & anaerobic ATP yield, inefficient
$w_O$ & & $\frac{g_Y}{mmol}$ & Oxygen yield
$c_S$ & &  $\frac{L}{g_Yh}$ & Maximum Sugar uptake, the fastest yeast processes sugar per unit biomass
$u_S$ & & $dimensionless$ & Monod equation describing instantaneous uptake of sugar
$c_O$ & & $\frac{L}{g_Yh}$ & Oxygen uptake, how quickly yeast processes Oxygen per unit biomass
\hline
\end{tabular}
\caption{Example table.}
\label{tab:example}
\end{table}

\subsection*{Uptake}
\par # FILL OUT

\subsection*{Decrease and Replenishment of Resources}
\par Ordinarily, and in our hyptothetical experiment, there is no baker's sugar added to the dough. The sugar present is tied up in the flour. As discussed in a previous section, when the flour gets wet and is kneaded, enzymes in the flour go to work breaking down the sugar--rather than all the sugar being available at once, it replenishes logistically. Therefore we must add a replenishing term to our change of sugar equation. The change in sugar equation becomes
$$\frac{dS}{dt} = r_S S (1 - \frac{S}{P_S}) - c_S Y S$$
where $\frac{dS}{dt}$ is the change in sugar over time in $\frac{GGE}{L \cdot h}$, $- u_S^{S}(S) Y$ is instantaneous amount of sugar metabolized by yeast, and $r_S S (1 - \frac{S}{P_S})$ is the logistic replenishment term, where $r_S$ is the logistic growth constant and $P_S$ is the total amount of sugar in the loaf, available or unavailable.  

\par **NOTE:** We should think about this. Is it logistic? Shouldn't the carrying capacity be changing because there is ever less sugar? If the yeast disappeared at some point, the sugar would no longer be approaching the same limit.

\par Since oxygen is not replenished, its change equation only has a consumption term.
$$\frac{dO}{dt} = -c_O O Y$$

\subsection*{Yield}
\par Additionally, the yield sugar provides yeast varies with the amount of oxygen available. When oxygen is plentiful, yeast metabolizes sugar aerobically, which is both quicker and creates more energy for the yeast to grow. When oxygen is low, yeast metabolizes the sugar less anaerobically, which is less efficient. Our model should reflect this--a relatively short burst of growth until oxygen is depleted, and then slow, anaerobic growth until the supply of sugar is exhausted. The anaerobic metabolism creates $CO_2$ and ethanol as byproducts, and is called fermentation. In our model, we want to include a smooth switch between aerobic and anaerobic metabolism. We accomplish this via a switching function, which we name $f$ in our model.
$$f(O) = \frac{O}{K_O + O}$$
\par Where $O$ is the amount of oxygen and $K_O$ is the half-saturation constant, where yeast behaves 50 aerobically and 50 anaerobically. The full effective yield term, which controls the yield that sugar provides the yeast for a given amount of oxygen, is the following.
$$w_S^{eff} = [w_S^{ana}​+(w_S^{aer}​−w_S^{ana}​)f(O)]$$
\par Thus, when $O \gg K_O$, $f(O) \approx 1$, and $w_S^{eff} \approx w_S^{aer}$ indicating fully aerobic yield. Likewise, when $O \ll K_O$, $f(O) \approx 0$, and $w_S^{eff} \approx w_S^ana$, indicating fully anaerobic yield. 

\subsection*{Byproduct}
\par The $CO_2$ created during this process exists outside the consumer resource relationship, and is simply stoichiometrically related to the rate of anaerobic sugar consumption. Since our $f(O)$ function gives the aerobic fraction of sugar metabolism, $1 - f(O)$ will be the anaerobic fraction. So the instantaneous rate of anaerobic sugar consumption is 

$$(1 - f(O))u_S(S)Y$$

\par And gives grams per Liter per Hour. Since, in this reaction, one gram becomes .0111 mols of $CO_2$, we can multiply this by .0111 to get instantaneous rate of production of CO_2, in mols per liter.

\subsection*{Final Form}
\par By assembling these pieces into the general MacArthur consumer resource model found above, we have the following ODE.

$$\frac{dY}{dt} = Y \bigg[w_O c_O O + [w_S^{ana}​+(w_S^{aer}​−w_S^{ana}​)f(O)]u_S(S) - m\bigg]$$

$$\frac{dO}{dt} = -c_O O Y$$

$$\frac{dS}{dt} = r_S S (1 - \frac{S}{P_S}) - c_S Y S$$

$$\frac{dC}{dt} = (1 - f(O))u_S(S)Y * 0.0111$$

%% Third Section
\section{Results}

Clearly and succinctly state and describe the conclusions that you can draw from the model you have achieved (or the many failed attempts). Does your model(s) perform well quantitatively or qualitatively?

%%Fourth Section
\section{Analysis/Conclusions}

Discuss the appropriateness of the techniques/methods you employed in modeling. Did your group appropriately model the chosen phenomenon? If not, what different steps could you have taken if you had more time? What did you learn about the techniques/method that were used in the group project? If your model was successful, what additional insight/conclusions could you obtain from it? For instance, if you had a successfully modified SIR model, how might it affect different government policy? If you had a successful model for the spread of inaccurate information on social media, how might it be implemented to help reduce the spread of inaccurate information?

\subsection*{MacArthur Consumer-Resource}
Unfortunately we were unable to find real-world data which we could use to evaluate this model quantitatively. Instead we will pick reasonable parameter values and evaluate the results qualitatively. 

\centering
\begin{tabular}{l c c c}
\hline
\textbf{Parameter} & \textbf{Value} & \textbf{units} & \textbf{Sources} \\ 
\hline
$Y_0$ & 5.3 & $\frac{g_Y}{L}$ &  \\
$O_0$ & 0.25 & $\frac{mmol}{L}$ &  \\
$S_0$ & 3-5 & $\frac{GGE}{L}$ &  \\
$C_0$ &  & $\frac{mol}{L}$ &   \\
$K_O$ & 0.01 - 0.03 & $\frac{mmol}{L}$ &  \\
$K_S$ & 0.1 - 0.5 & $\frac{GGE}{L}$ &  \\
$P_S$ & 10 - 20 & $\frac{g_S}{L}$ &  \\
$r_S$ &  & $\frac{1}{h}$ &  \\
$w_S^{aer}$ & 0.45 - 0.55 & $\frac{g_Y}{g_S}$ & \cite{aerobic_yield1} \cite{aerobic_yield2} \\
$w_S^{ana}$ & 0.1 & $\frac{g_Y}{g_S}$ & \cite{anaaerobic_yield1} \cite{anaerobic_yield2} \\
$w_O$ & 0.016 & $\frac{g_Y}{mmol}$ & \cite{02_yield1}
$c_S$ & &  $\frac{L}{g_Yh}$ & \\
$c_O$ & 0.1 & $\frac{L}{g_Yh}$ & \cite{o2_uptake} \\
\hline
\end{tabular}
\caption{Example table.}
\label{tab:example}
\end{table}





This part should all be done before you get to \emph{page 11}.  The bibliography can spill on to page 11, but we won't read text that goes past page 10.


%%%%%%%%%%%%%%%%%%%%%%%%%%%%%%%%%%%%%
%% Bibliography below
%%%%%%%%%%%%%%%%%%%%%%%%%%%%%%%%%%%%%
\FloatBarrier % Keep the figures from being put after the bibliography
\newpage
\bibliography{refs}
\bibliographystyle{alpha}

\end{document}
